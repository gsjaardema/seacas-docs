\chapter{Effect of Aspect Ratio}

\section{Introduction}

     An important issue for radioactive materials transportation casks
is the aspect ratio at which shallow angle slapdown impact events
result in a higher velocity secondary impact than primary impact.  Two
primary parameters have been identified, slenderness and location of
the center of gravity. 

\section{Slenderness}

The slenderness issue was studied using solid and hollow cylinders
(simple models which capture the essential geometry of most
radioactive materials transportation casks).  For the solid cylinder
model, the initial geometry had a radius of 30 and a length of 120.
The length was reduced in steps to 50 while the radius was increased
to keep the overall model volume constant. The
hollow cylinder model was treated similarly.  The initial 
hollow cylinder model
had a length of 120 with an outer radius of 30 and an inner
radius of 21.21.  These radii gave a cross-sectional area of the
hollow cylinder equal to half the area enclosed by the outer radius.
As for the solid cylinder, the length was reduced to 50 while the
overall model volume and the relationship between cross-sectional
areas were held constant. The details of the model geometries are
shown in Tables 4.1 and 4.3.  
The slapdown analysis was performed for an
initial angle of 15$^\circ$.  Two linear springs were investigated, one of
which was elastic and the other plastic.  The elastic spring had a
linear spring constant for both loading and unloading of 600,000. 
Thus no energy was absorbed in the spring.  The plastic spring had a
loading constant of 600,000 (same as the elastic spring) but had an
unloading constant of 600,000,000,000 (6 orders of magnitude higher
than the loading constant).  The effect of this high unloading
constant was to prevent energy being returned to the structure on
spring unloading thus giving an almost perfectly plastic spring. The
initial angle of 15$^\circ$ was sufficient to ensure that for all cases,
the nose was rebounding from the target prior to the tail impact.  No
frictional effects were considered here. 

\begin{table}
\begin{center}
\caption{Effect of Aspect Ratio on Slapdown for Solid Cylinder Model
with a Linear-Elastic Spring}
\makeqnum
\begin{tabular}{||r|r|r|r|r|r|r||}
\hline
&\multicolumn{1}{c|}{Outer}  
&\multicolumn{1}{c|}{Moment of} 
&\multicolumn{1}{c|}{Radius of} 
&\multicolumn{1}{c|}{Aspect} 
&\multicolumn{1}{c|}{Tail}
&\multicolumn{1}{c||}{Tail}\\
\multicolumn{1}{||c|}{Length} 
&\multicolumn{1}{c|}{Radius} 
&\multicolumn{1}{c|}{Inertia} 
&\multicolumn{1}{c|}{Gyration} 
&\multicolumn{1}{c|}{(L/r)}
&\multicolumn{1}{c|}{Velocity} 
&\multicolumn{1}{c||}{Displ}\\
\hline
 $120$ &$30.00$ &$??114,000$ &$?37.75$ &$3.18$ &$?-979$ &$6.028$\\
 $?80$ &$36.75$ &$???69,700$ &$?29.51$ &$2.71$ &$?-831$ &$5.689$\\
 $?60$ &$42.33$ &$???60,000$ &$?27.39$ &$2.19$ &$?-613$ &$4.745$\\
 $?55$ &$44.31$ &$???59,400$ &$?27.25$ &$2.02$ &$?-526$ &$4.243$\\
 $?50$ &$46.48$ &$???61,200$ &$?27.26$ &$1.83$ &$?-409$ &$3.468$\\
\hline
\end{tabular}
\end{center}

Analysis performed at the following initial conditions:

\makeqnum
\begin{tabular}{lll}
Initial vertical velocity &= &$-527.5$\\
Initial angle             &= &$???15.0^\circ$\\
Total mass                &= &$???80.0$\\
Spring - elastic     Load &= &$??600.0\times10^3$\\
Spring - elastic   Unload &= &$??600.0\times10^3$\\
\end{tabular}
\end{table}

\begin{table}
\begin{center}
\caption{Effect of Aspect Ratio on Slapdown for Solid Cylinder Model 
with a Plastic (Energy-Absorbing) Spring}
\makeqnum
\begin{tabular}{||r|r|r|r|r|r|r||}
\hline
&\multicolumn{1}{c|}{Outer}  
&\multicolumn{1}{c|}{Moment of} 
&\multicolumn{1}{c|}{Radius of} 
&\multicolumn{1}{c|}{Aspect} 
&\multicolumn{1}{c|}{Tail}
&\multicolumn{1}{c||}{Tail}\\
\multicolumn{1}{||c|}{Length} 
&\multicolumn{1}{c|}{Radius} 
&\multicolumn{1}{c|}{Inertia} 
&\multicolumn{1}{c|}{Gyration} 
&\multicolumn{1}{c|}{(L/r)}
&\multicolumn{1}{c|}{Velocity} 
&\multicolumn{1}{c||}{Displ}\\
\hline
 $120$ &$30.00$ &$??114,000$ &$?37.75$ &$3.18$ &$-754$ &$5.031$\\
 $?80$ &$36.75$ &$???69,700$ &$?29.51$ &$2.71$ &$-678$ &$4.804$\\
 $?60$ &$42.33$ &$???60,000$ &$?27.39$ &$2.19$ &$-567$ &$4.426$\\
 $?55$ &$44.31$ &$???59,400$ &$?27.25$ &$2.02$ &$-523$ &$4.261$\\
 $?50$ &$46.48$ &$???61,200$ &$?27.26$ &$1.83$ &$-465$ &$3.990$\\
\hline
\end{tabular}
\end{center}

Analysis performed at the following initial conditions:

\makeqnum
\begin{tabular}{lll}
Initial vertical velocity &=  &$-527.5$\\
Initial angle             &=  &$???15.0^\circ$\\
Total mass                &=  &$???80.0$\\
Spring - plastic     Load &=  &$??600.0\times10^3$\\
Spring - plastic   Unload &=  &$??600.0\times10^9$\\
\end{tabular}
\end{table}

\begin{figure}
\vspace{3.5 in}
\caption{Effect of Aspect Ratio (L/r) on Slapdown Severity}
\end{figure}

     The results of the slapdown analysis for the solid cylinders
with elastic and plastic springs
are presented in Tables 4.1 and 4.2 
respectively.  For the hollow cylinders, the results are shown 
in Tables 4.3 and 4.4. 
The ratio of length to
radius of gyration (L/r) was
selected to describe the slenderness of an object
subjected to shallow angle slapdown.
The vertical velocity of the tail
at the secondary impact and the maximum tail spring displacement 
were chosen to represent the severity of the secondary
impact event.  The maximum spring displacement is directly related to
the energy required to stop the tail of the object while the vertical
velocity at impact provides a clear representation of the slapdown
event independent of the tail spring characteristics. Tail vertical
velocity at impact, non-dimensionalized by the initial velocity, is
plotted against aspect ratio for both the solid and hollow
cylinders in Figure 4.1.  
In Figure 4.1, non-dimensional tail velocities
less than one, indicate that slapdown did not occur (secondary impact
was less severe than primary impact).  Slapdown did not occur when the
aspect ratio was less than 2 for both model geometries and both
spring types.  The plastic spring brings the tail
velocity at secondary impact closer to the initial velocity for all
aspect ratios.  Thus, with the plastic spring, secondary impact
velocities are lower for aspect ratios greater than 2 and higher 
(but still less than the initial velocity) for aspect ratios less
than 2.

\begin{table}
\begin{center}
\caption{Effect of Aspect Ratio on Slapdown for Hollow Cylinder Model 
with a Linear-Elastic Spring}
\makeqnum
\begin{tabular}{||r|r|r|r|r|r|r|r||}
\hline
&\multicolumn{1}{c|}{Outer}  
&\multicolumn{1}{c|}{Inner}
&\multicolumn{1}{c|}{Moment of} 
&\multicolumn{1}{c|}{Radius of} 
&\multicolumn{1}{c|}{Aspect}
&\multicolumn{1}{c|}{Tail}
&\multicolumn{1}{c||}{Tail}\\
\multicolumn{1}{||c|}{Length} 
&\multicolumn{1}{c|}{Radius} 
&\multicolumn{1}{c|}{Radius} 
&\multicolumn{1}{c|}{Inertia} 
&\multicolumn{1}{c|}{Gyration} 
&\multicolumn{1}{c|}{(L/r)}
&\multicolumn{1}{c|}{Velocity} 
&\multicolumn{1}{c||}{Displ}\\
\hline
$120$ &$30.00$ &$21.21$ &$??123,000$ &$?39.21$ &$3.06$ &$?-944$ 
&$5.974$\\
$?80$ &$36.75$ &$25.99$ &$???83,200$ &$?32.25$ &$2.48$ &$?-741$ 
&$5.360$\\
$?62$ &$41.74$ &$29.52$ &$???77,900$ &$?31.20$ &$1.99$ &$?-508$ 
&$4.128$\\
$?55$ &$44.31$ &$31.34$ &$???79,100$ &$?31.44$ &$1.75$ &$?-373$ 
&$3.202$\\
$?50$ &$46.48$ &$32.87$ &$???81,500$ &$?31.91$ &$1.57$ &$?-261$ 
&$2.315$\\
\hline
\end{tabular}
\end{center}

Analysis performed at the following initial conditions:

\makeqnum
\begin{tabular}{lll}
Initial vertical velocity &=  &$-527.5$\\
Initial angle             &=  &$???15.0^\circ$\\
Total mass                &=  &$???80.0$\\
Spring - elastic     Load &=  &$??600.0\times10^3$\\
Spring - elastic   Unload &=  &$??600.0\times10^3$\\
\end{tabular}
\end{table}

\begin{table}
\begin{center}
\caption{Effect of Aspect Ratio on Slapdown for Hollow Cylinder Model 
with a Plastic (Energy-Absorbing) Spring}
\makeqnum
\begin{tabular}{||r|r|r|r|r|r|r|r||}
\hline
&\multicolumn{1}{c|}{Outer}  
&\multicolumn{1}{c|}{Inner}
&\multicolumn{1}{c|}{Moment of} 
&\multicolumn{1}{c|}{Radius of} 
&\multicolumn{1}{c|}{Aspect}
&\multicolumn{1}{c|}{Tail}
&\multicolumn{1}{c||}{Tail}\\
\multicolumn{1}{||c|}{Length} 
&\multicolumn{1}{c|}{Radius} 
&\multicolumn{1}{c|}{Radius} 
&\multicolumn{1}{c|}{Inertia} 
&\multicolumn{1}{c|}{Gyration} 
&\multicolumn{1}{c|}{(L/r)}
&\multicolumn{1}{c|}{Velocity} 
&\multicolumn{1}{c||}{Displ}\\
\hline
$120$ &$30.00$ &$21.21$ &$??123,000$ &$?39.21$ &$3.06$ &$-736$ 
&$4.962$\\
$?80$ &$36.75$ &$25.99$ &$???83,200$ &$?32.25$ &$2.48$ &$-633$
&$4.659$\\
$?62$ &$41.74$ &$29.52$ &$???77,900$ &$?31.20$ &$1.99$ &$-515$ 
&$4.224$\\
$?55$ &$44.31$ &$31.34$ &$???79,100$ &$?31.44$ &$1.75$ &$-448$
&$3.896$\\
$?50$ &$46.48$ &$32.87$ &$???81,500$ &$?31.91$ &$1.57$ &$-391$ 
&$3.562$\\
\hline
\end{tabular}
\end{center}

Analysis performed at the following initial conditions:

\makeqnum
\begin{tabular}{lll}
Initial vertical velocity &=  &$-527.5$\\
Initial angle             &=  &$???15.0^\circ$\\
Total mass                &=  &$???80.0$\\
Spring - plastic     Load &=  &$??600.0\times10^3$\\
Spring - plastic   Unload &=  &$??600.0\times10^9$\\
\end{tabular}
\end{table}

     A secondary impact less severe than the primary impact has been
shown, in Chapter 2, 
to be a general relationship for aspect ratios less than two.  In 
Chapter 2, the aspect ratio was defined in terms of $\beta_{n} = 
l_{n}/r$ and $\beta_{t} = l_{t}/r$ where:
\begin{tabbing}
$r  \;$ \= is the radius of gyration,\\
$l_{n}$ \> is the distance from the center of mass to the nose, and\\
$l_{t}$ \> is the distance from the center of mass to the tail.\\
\end{tabbing}

Here $l_{n}$ = $l_{t}$ = ${{L}\over{2}}$, so that
\begin{equation}
\beta_{n} = \beta_{t} = {{L}\over{2r}}.
\end{equation}

Thus
\begin{equation}
\beta_{n}\cdot\beta_{t} = {{L^{2}}\over{4r^{2}}}.
\end{equation}

As shown in Equation 2.3.19, the tail impact velocity is less than the 
nose impact velocity when 
\begin{equation}
\beta_{n} \cdot \beta_{t}\leq 1
\end{equation}
or when
\begin{equation}
{{L}\over{r}} \leq 2.
\end{equation}

\section{Center of Gravity Location}

     In most
transportation systems, due to the impact limiters, closure
systems, gamma shielding, and contents, the center of gravity rarely
coincides with the midpoint between loading springs. Thus, 
variation in the location of the center of gravity 
are of concern for transportation system design.  Two of the solid
cylinder models, those with length 120 and length 55, were chosen to
investigate this phenomenon.  These two models provide a wide
variation in aspect ratio (3.18 and 2.02, respectively).  The
center of gravity was shifted independently of all other parameters
including the moment of inertia. A physical interpretation of this
shift is difficult to provide.  It can be regarded as resulting from
the appropriate distribution of a variable density material within the
confines of the solid cylinder model geometry.  Alternately, it can be
regarded as a structure with the given mass and moment of inertia
placed at varying locations along a rigid massless bar. The elastic
and plastic springs described in Section 4.2 were also used here. 

\begin{figure}
\vspace{3.5 in}
\caption{Effect of the Location of the center of gravity on Slapdown 
Severity}
\end{figure}

     The results of shifting the center of gravity axially on the two
solid cylinder models are shown in Figure 4.2 and in
Tables 4.5 and 4.6.  As can be seen
in these tables, both the aspect ratio and the nose spring
characteristics, influence 
the effects shifting the center of gravity.
For the long cylinder, the tail velocity is
maximized when the center of gravity is at the 30\% location (closer to
the nose than the tail) for both the linear and the nonlinear springs.
For the short cylinder, maximum tail velocity occurs when the 
center of gravity 
coincides with the cylinder center (50\% location). As
expected, for an arbitrary center of gravity location, maximum energy
absorption at the tail does not coincide with maximum tail
velocity.  The maximum tail energy occurs at a different 
center of gravity 
location for each case investigated.  For the long cylinder
with an elastic nose spring, maximum tail energy occurs at a 
center of gravity 
location of 60\% (slightly toward the tail).  With a plastic
nose spring, maximum tail energy occurs at the 90\% location. For the
short cylinder, the maximum tail energy center of gravity location is
70\% for an elastic nose spring and 80\% for a plastic nose spring.  

\begin{table}
\begin{center}
\caption{Effect of Shift of center of gravity on Slapdown for Solid
Cylinder Model Length=120 with Elastic and Plastic Spring}
\makeqnum
\begin{tabular}{||c|r|r|c|r|r||}
\hline
\multicolumn{3}{||c|}{Elastic Spring} &\multicolumn{3}{c||}{Plastic 
Spring}\\
\hline
\multicolumn{1}{||c|}{C. G.}
&  &
&\multicolumn{1}{c|}{C. G.}
& &\\
\multicolumn{1}{||c|}{Location}
&\multicolumn{1}{c|}{Tail}
&\multicolumn{1}{c|}{Tail}
&\multicolumn{1}{c|}{Location}
&\multicolumn{1}{c|}{Tail}
&\multicolumn{1}{c||}{Tail}\\
\multicolumn{1}{||c|}{\% Length}
&\multicolumn{1}{c|}{Velocity}
&\multicolumn{1}{c|}{Displ}
&\multicolumn{1}{c|}{\% Length}
&\multicolumn{1}{c|}{Velocity}
&\multicolumn{1}{c||}{Displ}\\
$10$ &$?-406$ &$1.557$ &$10$ &$-470$ &$1.791$\\
$20$ &$?-967$ &$4.085$ &$20$ &$-743$ &$3.165$\\
$30$ &$-1132$ &$5.357$ &$30$ &$-827$ &$4.080$\\
$40$ &$-1093$ &$5.865$ &$40$ &$-809$ &$4.654$\\
$50$ &$?-979$ &$6.028$ &$50$ &$-754$ &$5.031$\\
$60$ &$?-847$ &$6.065$ &$60$ &$-689$ &$5.263$\\
$70$ &$?-721$ &$6.049$ &$70$ &$-626$ &$5.466$\\
$80$ &$?-608$ &$5.945$ &$80$ &$-570$ &$5.607$\\
$90$ &$?-501$ &$5.613$ &$90$ &$-520$ &$5.722$\\
\hline
\end{tabular}
\end{center}

Analysis performed at the following initial conditions:

\makeqnum
\begin{tabular}{llllll}
Initial vertical velocity &= &$-527.5$
&Initial vertical velocity &= &$-527.5$\\
Initial angle &= &$???15.0^\circ$ &Initial angle &= &$???15.0^\circ$\\
Total mass &= &$???80.0$ &Total mass &= &$???80.0$\\
Spring - elastic     Load &=  &$??600.0\times10^3$
&Spring - plastic     Load &= &$??600.0\times10^3$\\
Spring - elastic   Unload &= &$??600.0\times10^9$
&Spring - plastic   Unload &= &$??600.0\times10^9$\\
\end{tabular}
\end{table}

\begin{table}
\begin{center}
\caption{Effect of Shift of center of gravity on Slapdown for Solid
Cylinder Model Length=55 with Elastic and Plastic Spring}
\makeqnum
\begin{tabular}{||c|r|r|c|r|r||}
\hline
\multicolumn{3}{||c|}{Elastic Spring} &\multicolumn{3}{c||}{Plastic 
Spring}\\
\hline
\multicolumn{1}{||c|}{C. G.}
& &
&\multicolumn{1}{c|}{C. G.}
& &\\
\multicolumn{1}{||c|}{Location}
&\multicolumn{1}{c|}{Tail}
&\multicolumn{1}{c|}{Tail}
&\multicolumn{1}{c|}{Location}
&\multicolumn{1}{c|}{Tail}
&\multicolumn{1}{c||}{Tail}\\
\multicolumn{1}{||c|}{\% Length}
&\multicolumn{1}{c|}{Velocity}
&\multicolumn{1}{c|}{Displ}
&\multicolumn{1}{c|}{\% Length}
&\multicolumn{1}{c|}{Velocity}
&\multicolumn{1}{c||}{Displ}\\
$10$ &No Hit &$0.???$ &$10$ &$-198$ &$1.110$\\
$20$ &$-193$ &$1.146$ &$20$ &$-358$ &$2.185$\\
$30$ &$-401$ &$2.658$ &$30$ &$-460$ &$3.075$\\
$40$ &$-500$ &$3.655$ &$40$ &$-510$ &$3.755$\\
$50$ &$-526$ &$4.243$ &$50$ &$-523$ &$4.261$\\
$60$ &$-508$ &$4.522$ &$60$ &$-516$ &$4.636$\\
$70$ &$-467$ &$4.562$ &$70$ &$-496$ &$4.902$\\
$80$ &$-416$ &$4.398$ &$80$ &$-471$ &$5.044$\\
$90$ &$-361$ &$4.048$ &$90$ &$-444$ &$5.031$\\
\hline
\end{tabular}
\end{center}

Analysis performed at the following initial conditions:

\makeqnum
\begin{tabular}{llllll}
Initial vertical velocity &= &$-527.5$
&Initial vertical velocity &= &$-527.5$\\
Initial angle &= &$???15.0^\circ$ &Initial angle &= &$???15.0^\circ$\\
Total mass &= &$???80.0$ &Total mass &= &$???80.0$\\
Spring - elastic     Load &=  &$??600.0\times10^3$
&Spring - plastic     Load &= &$??600.0\times10^3$\\
Spring - elastic   Unload &=  &$??600.0\times10^9$
&Spring - plastic   Unload &=  &$??600.0\times10^9$\\
\end{tabular}
\end{table}

\section{Conclusions}
The ratio of length to radius of gyration is the most appropriate 
parameter to describe the body geometry for shallow angle 
slapdown.  The conclusion reached in Chapter 2 for linear behavior,
that for a length to radius of gyration ratio less than two, no 
slapdown (increase in tail velocity at impact) can occur, has been
numerically confirmed for nonlinear spring behavior as well.
It is difficult to draw general conclusions concerning location of the 
center of gravity.  For 
off-center location of the center of gravity, a simple relationship
between tail velocity at impact and required energy absorption no 
longer exists.  The maximum tail velocity occurs at a different 
center of gravity location than does the maximum tail spring energy 
absorption.  Therefore, severity of the secondary (tail) impact is 
difficult to define.  Thus, off-center location of the 
center of gravity can have an important effect on slapdown but each
case must be investigated individually.
