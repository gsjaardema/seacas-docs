\documentstyle[12pt]{sreport}       
%\documentstyle[12pt]{tex$1520:sand}       % Use old TeX
\makeatletter
\def\SLAP{{\sf Slapdown}}
\begin{document}
\begin{titlepage}   
 \let\footnotesize\normalsize  % Local definitions to make \thanks produce
\let\thempfn{\fnsymbol\thefootnote}
%\def\thefn{\fnsymbol\thefootnote}
 \thispagestyle{plain}
 \setcounter{page}{1}
\hbox{}\hfill{\parbox[t]{3.0in}{SAND88\up{--}???C\\
\raggedright To be presented at ?\par}}
\null
\vfil
\begin{center}
\tracingmacros2
   {\Large\bf Numerical and Analytical Methods for Approximating the
Eccentric Impact Response (Slapdown) of Deformable
Bodies\footnote{This
work performed at Sandia National Laboratories
supported by the U.~S.~Dept. of Energy under contract number
DE-AC04-76DP00789.}\par} 
   \vskip .5truein                  % Vertical space after title.
   \large
   {G. D. Sjaardema and G. W. Wellman}\\      % Set author in \large size.
   {Applied Mechanics Division I}\\
   {Sandia National Laboratories}\\
   {Albuquerque, New Mexico\ \ 87185}
\end{center} \par
\vfil
\null
\large
\hbox to \hsize{\hfil Abstract\hfil}
\vskip 2ex
\normalsize\noindent
Analytical and numerical methods have been developed to approximate
the eccentric impact response of deformable bodies.  The eccentric
impact response is commonly known as {\em slapdown} since the
off-center impact gives the body a rotational velocity which causes
impact at the opposite end.  A code, \SLAP , has been written to
approximate the slapdown behavior of deformable bodies. The body is
idealized as a three degree-of-freedom system with nonlinear impact
behavior. 

Several parameters of interest to the analysis and design of laydown
weapons were studied to determine their effects on the secondary
impact velocities (slapdown).  Parameters studied were the aspect
ratio of the body, the stiffness of the initial impact, and the
coefficient of friction between the target and the body. Rules for
applying the results of scale model tests to full scale bodies were
developed and confirmed for nonlinear spring behavior. 
\vfil
\null
\end{titlepage}
\end{document} 
