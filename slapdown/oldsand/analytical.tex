\chapter{Analytical Prediction of Slapdown Magnitude}
\section{Introduction}

In this chapter, simple equations are developed that can economically be
used to conservatively estimate the ratio of the initial vertical
velocity of the cask to the velocity of the cask at the secondary impact
point and also find the critical impact angle $\theta_0$ which will
produce the largest secondary impact velocity. The only variables
required are geometric and mass quantities, the force-displacement
behavior of the impact event is not required. 

\section{Nomenclature}

This section defines the nomenclature that is used in the derivation
of the Slapdown equation.  
\begin{description}
\item[$y$,$\dot y$,$\ddot y$] Vertical displacement, velocity, and
acceleration.  Positive toward rigid surface,
\item[$\theta$,$\omega$,$\alpha$] Angular inclination, velocity, and
acceleration of the axis of the body with respect
to the rigid surface, positive CCW if the nose is to the left of the tail,
\item[$m$,$I$]   Body mass and Moment of inertia,
\item[$l$]       Length from the center of gravity to the impact point
measured along the axis of the body,
\item[$k$]       Spring constant (force/length),
\item[$v_0$,$\theta_0$]     Initial vertical velocity and angular
position of the body, 
\item[$\omega_n$] Natural frequency of the differential equation, and 
\item[$t$]        Time.
\end{description}
For the geometric variables and the linear displacement quantities, the
subscripts $(\cdot)_n$ and $(\cdot)_t$ designate the nose and tail 
of the body, respectively; a superposed dot indicates the time
derivative of the dotted quantity.

\section{Derivation of Slapdown Equation}

The cask is approximated by the three degree-of-freedom system shown in
Figure~\ref{f:gi} on page~\pageref{f:gi}.  The force-displacement
behavior is approximated by springs located at the initial and secondary
impact points. The slapdown equation is developed assuming that (1)~the
springs are linear elastic, (2)~at most one spring is in contact with
the rigid surface at any time, and (3)~there are no horizontal forces
due to friction between the rigid surface and the cask.  Assumption~(3)
reduces the body to a two degree-of-freedom system, and assumption~(2)
further reduces it to a sequence of one degree-of-freedom systems. It
will be shown that the linear elastic behavior is a conservative
assumption that maximizes the secondary impact velocity. 

In the text that follows, the nose and tail of the body refer to
the initial and secondary impact points, respectively.  The sign
convention used is that displacements, velocities and accelerations are
positive downward; and angular positions, velocities and accelerations
are positive clockwise. 

\subsection{Rigid Body Motion Equations}

Using D'Alembert's principle, the vertical acceleration $\ddot y_n$ at the
nose and the angular acceleration $\alpha$ about the nose during
the time interval that only the nose is in contact with the rigid
surface can be determined by the summation of moments and forces. 
\begin{eqnarray}
\sum M_n\rightarrow 0  & = & 2 \ddot y_n l_n +\alpha(l_n-r)^2
     +\alpha(l_n+r)^2\label{e1} \\ 
\sum F_n\rightarrow -ky_n & = & \frac{m}{2}\left\{\left[(l_n-r)\alpha 
     +\ddot y_n \right] +\left[(l_n+r)\alpha+\ddot y_n\right]\right\}\label{e2}
\end{eqnarray}
\begin{tabbing}
where: \=ZZ\= \kill
where: \>$r  $\> is the radius of gyration ${}=\sqrt{\frac{I}{m}}$,\\
       \>$l_n$\> is the distance from the center of mass to the nose,\\ 
       \>$y_n$\> is the vertical displacement of the nose,\\
       \>$k  $\> is the elastic spring constant, and\\
       \>     \> it is assumed that $\cos\theta\approx 1$.
\end{tabbing}
Equation~\eref{e1} can be used to give $\alpha$ in terms of $\ddot y_n$:
\begin{equation}
\alpha = \frac{-\ddot y_n l_n}{l_n^2 + r^2}\label{eq:alpha}
\end{equation}
Using this result, Equation~\eref{e2} can be written as the single degree of 
freedom equation:
\begin{equation}
\ddot y_n\left[\frac{m}{\beta_n^2+1}\right] + ky_n = 0\label{eq:diff}
\end{equation}
where $\beta_n$ is defined to be $l_n/r$.  Equation~\eref{eq:diff} has the
solution
\begin{eqnarray}
     y_n & = & A\cos\omega_n t + B\sin\omega_n t\label{eq:acbs}\\
\omega_n & = & \sqrt{\frac{k}{m}\left(\beta_n^2+1\right)}
\end{eqnarray}
At time $t=0$, $\dot y_n = v_0$ and $y_n = 0$.  Substituting these 
conditions into Equation~\eref{eq:acbs} results in the following equations
for the vertical displacement, velocity, and acceleration of the nose: 
\begin{eqnarray}
      y_n & = &  \frac{v_0}{\omega_n}\sin\omega_n t\\
 \dot y_n & = &  v_0\cos\omega_n t\label{eq:vl}\\
\ddot y_n & = & -v_0\omega_n\sin\omega_n t\label{eq:ayl}
\end{eqnarray}
The angular quantities can be determined by substituting 
Equation~\eref{eq:ayl} into Equation~\eref{eq:alpha} and integrating.
\begin{eqnarray}
\alpha & = & \frac{\beta_n v_0\omega_n}{r(\beta_n^2+1)}\sin\omega_n t\\
\omega & = & \frac{-\beta_n v_0}{r(\beta_n^2+1)}[\cos\omega_n t - 1]
               \label{eq:omega}\\
\theta & = & \frac{-\beta_n v_0}{r(\beta_n^2+1)}
                \left[\frac{\sin\omega_n t}{\omega_n} - t\right] + \theta_0
\label{etheta}
\end{eqnarray}

These results are only valid during the time interval ($0\leq t\leq \pi
/ \omega_n$) that the spring is in contact with the rigid surface.  For
times $t>\pi/\omega_n$ and before the tail contacts the rigid
surface, 
\begin{eqnarray}
\alpha & = & 0, \\
\omega & = & \frac{2\beta_n v_0}{r(\beta_n^2+1)},\\
\theta & = & \frac{\beta_n v_0}{r(\beta_n^2+1)}\left[\frac{\pi}{\omega_n}
      + 2t\right] + \theta_0\label{e12}
\end{eqnarray}

Equation~\eref{e12} can be used to determine the minimum initial drop angle
$\theta_0$ such that the secondary impact occurs at the same time or
after the initial impact point has rebounded from the rigid surface.
This angle must be adjusted if the body does not have the same radius at
the initial and secondary impact points. The angle
$\tan^{-1}\left(\frac{R_n-R_t}{l_n+l_t}\right)$ must be subtracted from
the initial angle $\theta_0$ to account for the nonsymmetric geometry.
$R_n$ and $R_t$ are the radii at the nose and tail of the body,
respectively, and the radius is defined as the perpendicular distance
from the longitudinal axis of the cask to the impact point.  

\subsection{Velocity of Tail at Impact}

The vertical velocity at the tail of the body $v_t$ is related to
the vertical velocity and angular velocity at the nose by 
\begin{equation}
v_t = v_n + \omega(l_n + l_t)\cos\theta
\end{equation}
where $v_n = \dot y_n$. 
Substituting Equations~\eref{eq:vl} and~\eref{eq:omega} into the above
equation and assuming that $\cos\theta\approx 1$, the ratio of the velocity
at the tail to the initial vertical velocity is:
\begin{equation}
\frac{v_t}{v_0} = \cos\omega_nt - \frac{\beta_n(\beta_n+\beta_t)}
{(1+\beta_n^2)}\left[\cos\omega_n t-1\right]\label{eq:vrv0t}
\end{equation}
At time $t = \pi/\omega_n$, the maximum velocity ratio is
\begin{equation}
\frac{v_t}{v_0} = \frac{2\beta_n(\beta_n+\beta_t)}{1+\beta_n^2} - 1
\label{eq:vrv0}
\end{equation}

This result indicates that if linear elastic response is assumed, the
magnitude of the slapdown velocity of the body can be estimated using
only geometric and mass quantities; the force-displacement behavior of
the body is not needed.  Equation~\eref{eq:vrv0} can be used to show that
\begin{equation}
{\rm If} \quad\beta_n\beta_t \leq1, \quad{\rm then} 
\quad\frac{v_t}{v_0} \leq1\label{eq:blbr}
\end{equation}

\subsection{Effect of Inelastic Force-Displacement Behavior}

The results in the previous section were derived assuming a linear
elastic force-displacement behavior.  In general, the force-displacement
behavior is inelastic and energy is dissipated by plastic deformation.
In this section it will be shown that when energy is dissipated by
plastic deformation, the slapdown velocity ratio is closer to 
unity than if there is no energy dissipation. 

In this derivation, the springs are assumed to have an infinite
unloading modulus which means that all of the internal energy in the
spring is dissipated.  Equations \eref{e1} through~\eref{etheta} are
still valid, except that the time interval is reduced to ($0\leq t\leq
\pi / 2\omega_n$).  If the tail impacts the rigid surface at time
$t=\pi/2\omega_n$, then Equation~\eref{eq:vrv0t} gives the following
expression for the maximum inelastic velocity ratio: 
\begin{equation}
\frac{v_t}{v_0} = \frac{\beta_n(\beta_n+\beta_t)}{1+\beta_n^2}
\label{eq:vrv0i}
\end{equation}
which can be written as:
\begin{equation}
\left(\frac{v_t}{v_0}\right)_\infty = \left(\frac{v_t}{v_0}\right)_E + 
  \left(\frac{1-\beta_n\beta_t}{1+\beta_n^2}\right)\label{eisum}
\end{equation}
where the subscripts $(\ )_\infty$ and $(\ )_E$ denote infinite
unloading modulus and elastic unloading modulus, respectively.  Using
the relation stated in Equation~\eref{eq:blbr}, the last term on the
right-hand-side of Equation~\eref{eisum} is greater than zero 
if $\beta_n\beta_t\leq1$ and is less than zero if $\beta_n\beta_t\geq1$.
Therefore, an infinite unloading modulus results in a tip velocity
ratio that is closer to unity than the velocity ratio calculated
assuming linear elastic impact behavior.  These two cases bound the
possible unloading behavior of all bodies, therefore
\begin{displaymath}
\begin{array}{cccc}
{\rm If\;} &
   \left\{\begin{array}{rcl}
       \beta_n\beta_t & \leq & 1\\
       {} & {} & {}\\
       \beta_n\beta_t & \geq & 1
   \end{array}\right\}, &
{\quad\rm then\quad} &
   \left\{\begin{array}{rclcl}
       \left[\frac{v_t}{v_0}\right]_E & \leq & 
                       \left[\frac{v_t}{v_0}\right]_I & \leq & 1\\
       {} & {} & {} & {} & {}\\
       \left[\frac{v_t}{v_0}\right]_E & \geq & 
                       \left[\frac{v_t}{v_0}\right]_I & \geq & 1
   \end{array}\right\}
\end{array}
\end{displaymath}
where the subscripts $(\ )_E$ and $(\ )_I$ refer to the elastic and
inelastic responses, respectively. The above results have been verified
using the \SLAP\ program. Note that although the velocity
ratio for the inelastic case increases when $\beta_n\beta_t \leq  1$, it
does not increase to greater than unity.  The effect of nonzero friction
forces is being investigated. 

\section{Conclusions}

An equation has been derived which gives a conservative estimate of the
velocity ratio for slapdown of a body impacting a rigid surface.  Only
easily obtainable geometric and mass quantities are required.  This
equation can be used during the design process to help determine a cask
geometry which will minimize the slapdown potential. It can also be used
to determine the approximate impact angle for a three-dimensional finite
element analysis of the slapdown event. 

