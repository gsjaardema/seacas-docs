\documentstyle[12pt]{tex$1520:sand}
\newcommand{\SLAP}{\mbox{\sf Slapdown}}
\begin{document}

\section{Mechanics}
\subsection{Equations of Motion}

The \SLAP\ program approximates the response of the continuum body with 
a three degree of freedom model.  The degrees of freedom are the rotation
of the body about the center of mass, and the vertical and horizontal 
displacement of the center of mass.  The deformations of the continuum 
are modeled with nonlinear springs at the points of contact of the body 
with the rigid surface.  The equations of motion for the model are:
\begin{equation}
M\ddot u + f^{int} = f^{ext}
\end{equation}
where $M$ is the mass matrix and $f^{int}$ and $f^{ext}$ are the internal 
and external forces, respectively.  The above equation can also be 
written as:
\begin{equation}
M\ddot u = f 
f = f^{ext} - f^{int}
\end{equation}

This equation can be written in matrix notation at:
\begin{equation} 
\left[ \begin{array}{ccc}
  J & 0 & 0 \\
  0 & m & 0 \\
  0 & 0 & m
\end{array} \right]
=
\left\{ \begin{array}{c}
  \theta \\
  u_y \\
  u_x
\end{array} \right\}

\subsection{Time Integration}
The \SLAP\ program uses a central-difference explicit time integration 
method to integrate the second order differential equations which result 
from the discretization of the momentum equations.  The equations that 



\end{document}
