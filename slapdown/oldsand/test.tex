\documentstyle[12pt,sequence,draft]{tex$1520:sand}
\input{my$tex:macro.tex}
\input{my$tex:select_page}
%
\def\SLAP{{\sf Slap\-down}}
\def\KW#1{{\sf #1}} 
\def\NEW#1{\marginalstar{\sf #1}}
\def\N#1{\rlap{$^{#1}$}}
\long\def\Q#1#2{\llap{$^{#1}$}#2}
\def\REQ{\sf REQ}
%%%%%
%      REVISION MARKING AND FONT STYLE 
%
\def\version{0}
\def\NEW#1#2{\ifnum#1=\version \marginalstar{\sf #2}\else #2\fi}
\def\N#1{\rlap{$^{#1}$}}
%
%%%%%%
\makeatletter
\def\@oddhead{\ifnum\version<99 
      \small\sf DRAFT\hfil\filename, \Today, \Time\hfil REV:\version\fi}
\let\@evenhead\@oddhead
\let\@evenfoot\@oddfoot
\makeatother
\begin{document}
\begin{table}
\caption{Keywords for \SLAP---Geometric}\label{tgeom}\bigskip
\tabskip=1em plus 2em minus.5em
\hrule
\halign to \hsize {#\hfil&\hfil#\hfil&\vtop{\parindent=0pt\hsize=4.5in
  \hangindent1.5em  \strut#\strut}\cr
\em Keyword & \em Default\N1 & \hfill\em Description\hfill\cr
\noalign{\hrule}\medskip
\KW{HELP}  & --- & Lists all keywords and their values.
Required quantities that have not yet been assigned values are given the
value \hbox{\tt -9.999E0}.\cr
\noalign{\smallskip}
\KW{NOSE}  & --- & Indicates that the geometric
quantities following this line refer to the nose end of the body.  
If \KW{SYMMETRIC} is the second field, the geometric quantities that
have been defined for the tail are copied to the nose quantities. If
\KW{SQUARE} is the second field, the nose is treated as square (See
Figure~\ref{f:square}).\cr 
\noalign{\smallskip}
\KW{TAIL}  & --- & Indicates that the geometric
quantities following this line refer to the tail end of the body.
If \KW{SYMMETRIC} is the second field, the geometric quantities that
have been defined for the nose are copied to the tail
quantities.\cr 
\noalign{\smallskip}
\KW{LENGTH} & \REQ & The length along the
longitudinal axis from the center of gravity to a point perpendicular to
the contact point of the current end of the body. This distance is
designated as $Z_N$ and $Z_T$ in Figure~\ref{f:geom}. \cr 
\noalign{\smallskip}
\KW{RADIUS} & \REQ & The distance from the
longitudinal axis to the contact point of the current end of the body
measured perpendicular to the longitudinal axis. This distance is
designated as $R_N$ and $R_T$ in Figure~\ref{f:geom}. \cr
\noalign{\smallskip}
\KW{MU}     & 0.0 & Coefficient of Friction between body
and rigid surface at contact point\cr 
\noalign{\smallskip}
\KW{SPRING} & \REQ & Indicates the beginning of the spring
definition, spring data are input as pairs of data with one pair per
record.  Each record consists of the displacement and the force exerted
by the spring at that displacement.  Terminate the spring definition with 
\KW{END}\cr 
\noalign{\smallskip}
\KW{UNLOAD} & 0.0 & Spring unloading modulus.  If this
value is zero, the spring unloads elastically along the
force-displacement curve input for the spring. \cr}\medskip\hrule
NOTE:\\
\Q1{{`\REQ'} Indicates that a value must be input for this quantity; no
default is supplied. {\sf `---'} Indicates that this field is not
applicable.}
\end{table}
\end{document}
