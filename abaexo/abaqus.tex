\chapter{The \caps{ABAQUS} Output File Format} \label{appx:abaqus}

The \caps{ABAQUS} file format is described in the \caps{ABAQUS} Manual
Section 10. However, the manual is unclear as to the order of the record
types. This appendix defines the assumed record order determined by
examining actual \caps{ABAQUS} output files. These assumptions are
incorporated in \caps{\PROGRAM}, and if incorrect they may involve
program changes.

If the program encounters a record of an unexpected type, a warning is
printed and the record is ignored. Records that are out of order may
confuse the program.

The header records are in the following order:
\setlength{\itemsep}{\medskipamount} \begin{itemize}
\item a version record (1921) and a heading record (1922);
\item a series of element connectivity records (1900), which may not be
in numerical order;
\item a series of nodal coordinate records (1901), which may not be in
numerical order;
\item a start time steps record (1902), which is ignored; and
\item an end time step record (2001), which is ignored.
\end{itemize}

A series of time steps follow. The records for a time step are in the
following order:
\setlength{\itemsep}{\medskipamount} \begin{itemize}
\item a times record (2000) or an eigen record (?);
\item an output set request record (1911), which is ignored;
\item a series of element variable records (1..100), which may not be in
numerical order, but all variables for one element are together;
\item a series of nodal variable records (101..1000), which may not be in
numerical order, and all variables for a node may not be together;
\item an end time step record (2001), which is ignored.
\end{itemize}
