\input{titlepage.tex}
\section{GEN3D}
GEN3D transforms a 2D finite element geometry database into a 3D database.

SPONSOR:  Gregory D. Sjaardema, 1562 (505) 844-2701

KEYWORDS: preprocessor, mesh\_generation
\subsection{Documentation  }
``GEN3D - A GENESIS Database 2D to 3D Transformation Program''
Amy P. Gilkey and Gregory D. Sjaardema, SAND89-0485, 
March 1989.

``Updates to the mesh generation program GEN3D,''
Gregory D. Sjaardema, 1521, dated April 11, 1990.

\subsection{Unix}
\begin{verbatim}
     gen3d [-help] [-options option] exo_in exo_out
 Options:
     -executable=exe  Specifies alternate executable
     -help            Prints short usage summary
     -MANUAL          Prints extended usage summary
     -VMS             Reads/Writes files in VMS format (Unicos only)
\end{verbatim}
\begin{verbatim}
 Path:
     The standard location for gen3d on Unix systems is:
        /usr/local/eng_sci/struct/etc/gen3d
     See the entry 'seacas usage' for an up-to-date list of SEACAS
     locations for all supported systems.
\end{verbatim}
\subsection{VMS}
To execute GEN3D on VMS, type:

\begin{verbatim}
     GEN3D 2d_database 3d_database [input_file]
\end{verbatim}

``2d\_database'' is the filename of the input  2D GENESIS database.
``3d\_database'' is the filename of the output 3D GENESIS database.
``input\_file''  is the optional input file containing commands.

Commands are read from the terminal keyboard if ``input\_file''
is not specified. 
\subsection{Commands}
Mesh Transformation
\begin{verbatim}
   TRANSLATE  ROTATE     WARP_P     WARP_A     SPLINE    PROJECT
   SPLINE     PROJECT    TWIST      INTERVAL   SCALE     CHANGE    
\end{verbatim}

Mesh Orientation
\begin{verbatim}
   REVOLVE    REVCEN     OFFSET     MIRROR     ZERO      SHIFT
\end{verbatim}

Element Block Types
\begin{verbatim}
   BLOCK      TUNNEL     CENTER     NSETS      SSETS     CHANGE
\end{verbatim}

Information and Processing
\begin{verbatim}
   SHOW       LIST       HELP       END        QUIT      
\end{verbatim}

Undocumented and experimental
\begin{verbatim}
   TRANSPLINE
\end{verbatim}
\subsubsection{TRANSLATE}
\begin{verbatim}
 TRANSLATE {ntran} <1>, {tottran} <1.0>, {grad} <1.0>, ...
\end{verbatim}

TRANSLATE causes the 2D mesh to be translated to create the 3D mesh.
The number of levels is \{ntran\}, which is also the number of 3D elements
derived from each input 2D element.  The total range of the Z coordinate
is \{tottran\} with a gradient of \{grad\}.  The translation is always in
the negative Z direction.  This command supersedes previous
transformation commands.
The gradient affects the spacing of the levels.  The displacement or
thickness of level i is zi where:  
\begin{verbatim}
         z1 = tottran X grad - 1)/(grad^{ntran} - 1)  if grad ne 1;
              tottran / ntran                         if grad = 1
         zi = z1 X grad^{i-1}
\end{verbatim}
Multiple translation increments can be specified with a single translate
command by repeating the \{ntran\}, \{tottran\}, and \{grad\} parameters on
the command line.  For example, the following command specifies two
translation increments of thickness 1.0 for a total translation of 2.0: 

\begin{verbatim}
        TRANSLATE 5 1.0 0.5, 5 1.0 2.0
\end{verbatim}

All increments must be specified with a single TRANSLATE command.
\subsubsection{ROTATE}
\begin{verbatim}
 ROTATE {nrot} <1>, {totdeg} <360.0>, {grad} <1.0>, {cenrot} <0.0>
\end{verbatim}

ROTATE causes the 2D mesh to be rotated to create the 3D mesh.  The
number of rotation levels is \{nrot\}, which is also the number of 3D
elements derived from each input 2D element (with the exception of those
affected by the CENTER command).  The mesh is rotated a total of \{nrot\}
rotations through a total arc of \{totdeg\} degrees.  The angle of each
rotation is equal to \{grad\} times the previous rotation.  The center of
rotation \{cenrot\} and the gradient \{grad\} are only meaningful if no
center element blocks are defined (see the CENTER command).
\paragraph{Gradient}
The gradient affects the rotation of the levels.  The angular rotation
of level i is thetai where:  
\begin{verbatim}
         theta1 = totdeg X (grad - 1)/(grad^{nrot} - 1)  if grad ne 1;
                  totdeg / nrot                          if grad = 1
         thetai = theta1 X grad^{i-1}
\end{verbatim}
Rotation is always counterclockwise.  This command supersedes previous
transformation commands. 
\subsubsection{WARP\_P}
\begin{verbatim}
 WARP POINT {ntran} <1>, {tottran} <1.0>, {grad} <1.0>, {radius} <no
     default>, {edge_type} <RADIAL>
\end{verbatim}

WARP POINT causes the 2D mesh to be mapped onto a spherical surface to
create the 3D mesh.  The spherical surface has a radius of curvature
equal to \{radius\}.  The center of curvature is located on the z-axis,
and it is a distance of \{radius\} above the X-Y plane.  The number of
levels is \{ntran\}, which is also the number of 3D elements derived from
each input 2D element.  The total thickness (measured radially) is
\{tottran\} with a gradient of \{grad\}.  Note that \{radius\} must be greater
than the maximum distance from the z-axis to the boundary of the 2D
mesh.  This command supersedes previous transformation commands.
The \{edge\_type\}, which can be either VERTICAL or RADIAL, determines how
the created nodes are generated.  If VERTICAL is selected, the x and y
coordinates of the generated nodes are equal to the x and y coordinates
of the original 2D node.  If RADIAL is selected, the x and y coordinates
of the generated nodes are calculated to lie on a line from the center
of curvature (0.0, 0.0, \{radius\}) to the coordinates of the warped node
(xw, yw, zw) where xw and yw are the coordinates of the original 2D
node, and zw is determined such that the distance from the warped node
to the center of curvature is equal to \{radius\}.  Figure 2.1 in manual
illustrates the VERTICAL edge type, and Figure 2.2 in manual illustrates
the RADIAL edge type. 

The mesh transformation is performed in two parts.  First, the warped
nodal positions (xw, yw, zw) are calculated by mapping the original 2D
mesh onto a spherical surface with a radius of curvature equal to
\{radius\}.  The original x and y coordinates of the 2D mesh remain at the
same values; the z coordinate is calculated such that the distance to
the center of curvature is equal to \{radius\}.  
\begin{verbatim}
         xw = x0
         yw = y0
         zw = radius - (radius^2 - x0^2 - y0^2)^{1/2}
\end{verbatim}
The warped nodal positions are projections parallel to the z-axis onto a
spherical surface of radius \{radius\}; Figure 2.1 in manual illustrates
this process. Then, the generated nodal positions are determined by
translating either vertically or radially from the warped nodal
position.  A total of \{ntran\} translations are performed through a
distance of \{tottran\} with a gradient of \{grad\}.  Note that the
thickness is measured radially for either \{edge\_type\}. 

The gradient affects the spacing of the levels.  The thickness or length
of level i is zi where:  
\begin{verbatim}
         z1 = tottran X grad - 1)/(grad^{ntran} - 1)  if grad ne 1;
              tottran / ntran                         if grad = 1
         zi = z1 X grad^{i-1}
\end{verbatim}
\subsubsection{WARP\_A}
\begin{verbatim}
 WARP {axis} <no default>, {ntran} <1>, {tottran} <1.0>, {grad} <1.0>,
     {radius} <no default>, {edge_type} <RADIAL>
\end{verbatim}

This second form of the WARP command maps the 2D mesh to a cylindrical
surface centered on the \{axis\}-axis to create the 3D mesh.  The \{axis\}
parameter must be either X or Y.  The cylindrical surface has a radius
of curvature equal to \{radius\}.  The center of curvature is located a
distance of \{radius\} above the X-Y plane.  The number of levels is
\{ntran\}, which is also the number of 3D elements derived from each input
2D element.  The total thickness (measured radially) is \{tottran\} with a
gradient of \{grad\}.  This command supersedes previous transformation
commands.
The \{edge\_type\}, which can be either VERTICAL or RADIAL, determines how
the created nodes are generated.  If VERTICAL is selected, the x and y
coordinates of the generated nodes are equal to the x and y coordinates
of the projected 2D node.  If RADIAL is selected, the x and y
coordinates of the generated nodes are calculated to lie on a line from
the center of curvature to the coordinates of the warped node (xw, yw,
zw) where xw, yw, and zw are the coordinates of the mapped 2D node.

The mesh transformation is performed in two parts.  First, the warped
nodal positions (xw, yw, zw) are calculated by mapping the original 2D
mesh onto a cylinder about the \{axis\}-axis with a radius of curvature
equal to \{radius\}.  If \{axis\} is X, then the original X-coordinate
remains at the same value.  The generated Y and z coordinates are
calculated such that the distance from the generated node to the X-Z
plane measured along the cylindrical surface is equal to the X
coordinate of the node in the 2D mesh.  This is illustrated in Figure
2.2 in manual.  If \{axis\} is Y, the X's and Y's are switched in the
above discussion.  Then, the generated nodal positions are determined by
translating either vertically or radially from the warped nodal
position.  A total of \{ntran\} translations are performed through a
distance of \{tottran\} with a gradient of \{grad\}.  Note that the distance
is measured radially for either \{edge\_type\}. 

The gradient affects the spacing of the levels.  The thickness or length
of level i is zi where:  
\begin{verbatim}
         z1 = tottran X grad - 1)/(grad^{ntran} - 1)  if grad ne 1;
              tottran / ntran                         if grad = 1
         zi = z1 X grad^{i-1}
\end{verbatim}

The resulting 3D mesh will have an cylindrical angle of (x\{max\}/radius)
radians if warped about the Y axis, or (y\{max\}/radius) radians if warped
about the X axis, where x\{max\} and y\{max\} are the maximum X and Y
coordinates in the 2D mesh. 
\subsubsection{SPLINE}
\begin{verbatim}
 SPLINE {ntran} <1>, {tottran} <1.0>, {grad}  <1.0>,  ...
\end{verbatim}

where \{ntran\} is the number of levels, \{tottran\} is the total
transformation distance (thickness), and \{grad\} is the gradient which
affects the spacing of the levels (see page 21 in Reference GEN3D
for an explanation of the gradient).  Multiple transformation increments
can be specified with a single SPLINE command by repeating the \{ntran\},
\{tottran\}, \{grad\} parameters on a single line.  Note that the actual
thickness of the generated mesh is determined by the input front and
back spline surfaces; therefore, the distances entered as \{tottran\} are
the proportional distance of the segments.  For example, if the
following command was entered:

\begin{verbatim}
        SPLINE 2 1.0 0.5   4 2.0 1.0   2 1.0 2.0
\end{verbatim}

then, segments 1 and 3 would each be 25\% of the total thickness, and
segment 2 would be 50\% of the total thickness.

Following the SPLINE command line, GEN3D enters the spline input mode in
which the various spline options described below can be entered.
\paragraph{LINEAR}
\begin{verbatim}
 LINEAR
\end{verbatim}

the spline data are input as Radius-Z data pairs, and the slopes at the
end of the curves are linear slopes.
\paragraph{ANGULAR}
\begin{verbatim}
 ANGULAR
\end{verbatim}

the spline data are input as Theta(degrees)-Distance data pairs, where
Theta is the angle of the line between the origin (X = Y = Z = 0) and
the defined point and the Distance is the length of this curve.  The
slopes at the end of the curves are relative to this curve.  
\paragraph{ANGULAR}
\begin{verbatim}
 FRONT !  FRONT
\end{verbatim}

the curve data and slope specifications up to the next BACK, END, or
EXIT command will  the FRONT spline.  The front surface Z
values are greater (more positive) than the back surface Z values.
\paragraph{BACK}
\begin{verbatim}
 BACK
\end{verbatim}

the curve data and slope specifications up to the next FRONT, END, or
EXIT command will  the BACK spline.  The front surface Z
values are greater (more positive) than the back surface Z values.
\paragraph{LEFT}
\begin{verbatim}
 LEFT {slope}
\end{verbatim}

the parameter \{slope\} specifies the slope of the spline curve at the
LEFT end of the curve.  The slope is measured in the same units
specified in the ANGULAR or LINEAR command.  If the slope is not
specified, the end conditions of the curve will be set such that the
second derivative is equal to zero which is the so-called natural
cubic spline.
\paragraph{RIGHT}
\begin{verbatim}
 RIGHT {slope}
\end{verbatim}

the parameter \{slope\} specifies the slope of the spline curve at the
RIGHT end of the curve.  The slope is measured in the same units
specified in the ANGULAR or LINEAR command.  If the slope is not
specified, the end conditions of the curve will be set such that the
second derivative is equal to zero which is the so-called em natural
cubic spline.
\paragraph{EXIT}
\begin{verbatim}
 EXIT or END
\end{verbatim}

terminate spline input mode and return to general GEN3D command
processing.
\paragraph{See Also}
\begin{verbatim}
 Memo:  "Updates to the mesh generation program GEN3D,"
        Gregory D. Sjaardema, 1521, dated April 11, 1990.
        See the author for copies.
\end{verbatim}
\subsubsection{PROJECT}
\begin{verbatim}
 PROJECT {ntran} <1>, {tottran} <1.0>, {grad}  <1.0>, ...
\end{verbatim}

where \{ntran\} is the number of levels, \{tottran\} is the total
transformation distance (thickness), and \{grad\} is the gradient which
affects the spacing of the levels (see page 21 in Reference GEN3D
for an explanation of the gradient).  Multiple transformation increments
can be specified with a single PROJECT command by repeating the \{ntran\},
\{tottran\}, \{grad\} parameters on a single line.
\paragraph{NORMAL}
\begin{verbatim}
 NORMAL {x_normal} <0.0>, {y_normal} <0.0>, {z_normal} <0.0>
 PLANE  {x_normal} <0.0>, {y_normal} <0.0>, {z_normal} <0.0>
\end{verbatim}

NORMAL or PLANE  the normal vector to the back surface.  The
\{x\_normal\}, \{y\_normal\}, and \{z\_normal\} parameters are vector components
of the normal vector.  The front surface will be projected onto the
plane with the specified normal vector.  Because of the way GEN3D
generates the ddd mesh, the \{z\_normal\} component of the vector must be
negative; if a positive value of \{z\_normal\} is entered, all of the
components will be multiplied by negative one.footnoteThe test version
of GEN3D did not do this reversal correctly; instead it always made
\{z\_normal\} negative which made it confusing to determine the surface
orientation.  This bug has been fixed; however, old input files will now
give an incorrect orientation.  To use your old input files, either
correct the normal vector, or put the keyword DOOLDWAY after the
\{z\_normal\} component.

NORMAL or PLANE supersede previous WARP commands.
\paragraph{WARP}
\begin{verbatim}
 WARP {distance} {CONVEX|CONCAVE}
\end{verbatim}

WARP projects the front surface (original dd mesh) onto a spherical
surface with a radius of \{distance\}.  If CONVEX is specified, the
generated mesh will have a spherical or bulbous surface; if CONCAVE is
specified, the generated mesh will have a dimpled surface.  The distance
\{tottran\} specified in the PROJECT command is measured to the center of
the spherical surface.

WARP supersedes previous NORMAL or PLANE commands.  Note that the SPLINE
transformation can be used if a non-spherical surface is required.
\paragraph{SCALE}
\begin{verbatim}
 SCALE {x_scale} <1.0>, {y_scale} <1.0>
\end{verbatim}

SCALE multiplies the X and Y coordinates of the projected
surface by the respective scale factors.  See the SCLCEN command for the
equations used to transform the coordinates.
\paragraph{SCLCEN}
\begin{verbatim}
 SCLCEN {x_cen} <0.0>, {y_cen} <0.0>
\end{verbatim}

SCLCEN specifies the origin of the coordinate system for scaling.  This
is best illustrated by examining the equations used in the
transformation:  

\begin{verbatim}
       x_new = (x_old - x_cen) X x_scale + x_cen
       y_new = (y_old - y_cen) X y_scale + y_cen
\end{verbatim}

\paragraph{OFFSET}
\begin{verbatim}
 OFFSET {x_offset} <0.0>, {y_offset} <0.0>
\end{verbatim}

OFFSET offsets the nodal coordinates of the projected surface by the
specified \{x\_offset\} and \{y\_offset\} which shifts the back surface with
respect to the front surface.  The offsets are performed after the
surface has been projected onto the plane or spherical surface.
\paragraph{RESET}
\begin{verbatim}
 RESET
\end{verbatim}

RESET resets all parameters to their default values.
\paragraph{EXIT}
\begin{verbatim}
 EXIT or END
\end{verbatim}

EXIT or END terminates the PROJECT input mode and returns to normal GEN3D
command processing.
\paragraph{See Also}
\begin{verbatim}
 Memo:  "Updates to the mesh generation program GEN3D,"
        Gregory D. Sjaardema, 1521, dated April 11, 1990.
        See the author for copies.
\end{verbatim}
\subsubsection{TWIST}
\begin{verbatim}
 TWIST  {twangl} <0.0>, {twxcen} <0.0>, {twycen} <0.0>, {TRANSLATE},
        {ntran} <1>, {tottran} <1.0>, {grad} <1.0>,  ...      
\end{verbatim}

\begin{verbatim}
 TWIST  {twangl} <0.0>, {twxcen} <0.0>, {twycen} <0.0>, {Rotate}, 
        {cenrot} <0.0>, {ntran} <1>, {tottran} <1.0>, {grad}  <1.0>,  ...     
\end{verbatim}

where \{twangl\} is the rotational offset in degrees of the front surface
with respect to the back surface, \{twxcen\} and \{twxcen\} specify the
center of the twist rotation, \{ntran\} is the number of levels, \{tottran\}
is the total transformation distance (distance for the TRANSLATE option,
or degrees for the ROTATE option), and \{grad\} is the gradient which
affects the spacing of the levels (see page 21 in Reference GEN3D
for an explanation of the gradient).  Multiple transformation increments
can be specified by repeating the \{ntran\}, \{tottran\}, \{grad\} parameters
on a single line.

The twist angle per level is determined by dividing \{twangl\} by the
total number of translation increments \{ntran\}, not by the total
translation/rotation distance \{tottran\}.  Note that this means that the
incremental twist per level depends on the gradient.
\subsubsection{INTERVAL}
\begin{verbatim}
 INTERVAL {ntran} <1>, {tottran} <1.0>, {grad}  <1.0>, ...
\end{verbatim}

where \{ntran\} is the number of levels, \{tottran\} is the total
transformation distance in units applicable to the currently active
transformation option, and \{grad\} is the gradient which affects the
spacing of the levels (see page 21 in Reference GEN3D for an
explanation of the gradient).  Multiple transformation increments can be
specified with a single INTERVAL command by repeating the \{ntran\},
\{tottran\}, \{grad\} parameters on a single line.
\subsubsection{SCALE}
\begin{verbatim}
 SCALE {axis_1}, {scale_1}, {...} 
 SCALE ALL {scale} 
 SCALE RESET
\end{verbatim}

where \{axis\_i\} is either X, Y, or Z; and \{scale\_i\} is the factor by
which the specified axis coordinates are multiplied.  If ALL is
specified, all three coordinates are multiplied by \{scale\}.  Scaling can
be reset by the command SCALE RESET.
\subsubsection{CHANGE}
\begin{verbatim}
 CHANGE {Material|Nodeset|Sideset} {old_id} {new_id}
\end{verbatim}

where \{old\_id\} is an existing ID in the original mesh, and \{new\_id\} is
an ID that does not exist in the original mesh.  Note that you cannot
combine IDs using this command; the ID \{new\_id\} must not exist in the
original mesh, and it must not match an ID created using the
NODESET$|$SIDESET FRONT$|$BACK commands or the TUNNEL commands.

\subsubsection{REVOLVE}
\begin{verbatim}
 REVOLVE {axis_{1}}, {ndeg_{1}}, {axis_{2}}, {ndeg_{2}}, ... <last
      selection> REVOLVE RESET <initial condition>
\end{verbatim}

REVOLVE causes the transformed 3D mesh to be rotated.  Each (\{axis\},
\{ndeg\}) parameter pair specifies an axis (X or Y or Z) and the number of
degrees to rotate.  The axis refers to the ``viewing'' axis, not to the
object axis.  The rotations are according to right-hand rule.  The
center of the rotation is specified by the REVCEN command.

Revolutions are cumulative; however, only one center of revolution may
be specified.  The REVOLVE RESET command resets to no rotation.
\subsubsection{REVCEN}
\begin{verbatim}
 REVCEN {xcen} <2D minimum X coordinate>, {ycen} <2D minimum Y
      coordinate>, {zcen} <0.0>
\end{verbatim}

REVCEN sets the center of revolution for the REVOLVE command to the
point (\{xcen\},\{ycen\},\{zcen\}).
\subsubsection{SHIFT}
\begin{verbatim}
 SHIFT [ADD] {axis_{1}}, {offset_{1}}, {axis_{2}}, {offset_{2}}, ... 
 SHIFT ALL {offset} <0.0> 
 SHIFT RESET <initial condition> 
 SHIFT {xoff} <0.0>, {yoff} <0.0>, {zoff} <0.0>
\end{verbatim}

(NOTE: SHIFT is a synonym for OFFSET)
SHIFT specifies offsets to be added to the coordinates.  If a REVOLVE
command has been issued, the mesh is rotated before it is offset.  The
last form of the offset command is included to maintain compatibility
with the offset command in GEN3D.

SHIFT ALL offsets all of the coordinates by the specified offset, and
SHIFT RESET resets the offsets to zero.

Offsets are not cumulative unless ADD is specified, that is, 
if SHIFT X 0.5 X 1.0 is entered, the X coordinates will be offset by 1.0.
if SHIFT ADD X 0.5 X 1.0 is entered, the X coordinates will be offset by 1.5.
\subsubsection{OFFSET}
\begin{verbatim}
 OFFSET [ADD] {axis_{1}}, {offset_{1}}, {axis_{2}}, {offset_{2}}, ... 
 OFFSET ALL {offset} <0.0> 
 OFFSET RESET <initial condition> 
 OFFSET {xoff} <0.0>, {yoff} <0.0>, {zoff} <0.0>
\end{verbatim}

(NOTE: OFFSET is a synonym for SHIFT)
OFFSET specifies offsets to be added to the coordinates.  If a REVOLVE
command has been issued, the mesh is rotated before it is offset.  The
last form of the offset command is included to maintain compatibility
with the offset command in GEN3D.

OFFSET ALL offsets all of the coordinates by the specified offset, and
OFFSET RESET resets the offsets to zero.

Offsets are not cumulative unless ADD is specified, that is, 
if OFFSET X 0.5 X 1.0 is entered, the X coordinates will be offset by 1.0.
if OFFSET ADD X 0.5 X 1.0 is entered, the X coordinates will be offset by 1.5.
\subsubsection{MIRROR}
\begin{verbatim}
 MIRROR {axis_{1}}, {axis_{2}}, ... <no default> 
 MIRROR RESET <no reflections>
\end{verbatim}

MIRROR causes the transformed 3D mesh to be reflected about a coordinate
plane.  Each \{axis\} parameter specifies an axis (X or Y or Z) which is
the normal to the reflection plane.  Reflections are performed after the
mesh has been repositioned by the REVOLVE and OFFSET commands.

The MIRROR RESET command resets to no reflection.

Reflections are not cumulative, that is, if MIRROR X Y X is entered,
only one reflection about the X axis will be performed.  If an odd
number of reflections are performed, the element connectivity and the
sideset face numberings will be correctly reordered.
\subsubsection{ZERO}
\begin{verbatim}
 ZERO {axis_1}, {min_1}, {axis_2}, {min_2}, ... 
 ZERO RESET <no automatic zeroing>
\end{verbatim}

ZERO sets all \{axis\_i\} coordinates with an absolute value less than
\{min\_i\} equal to zero.  The ZERO RESET command resets to no automatic
zeroing.  This command is used to zero nodal coordinates that should be
equal to zero, but due to roundoff errors they have slightly nonzero
values.
\subsubsection{BLOCK}
\begin{verbatim}
 BLOCK {block_id_{1}}, {block_id_{2}}, ... <all element blocks>
\end{verbatim}

BLOCK defines the specified element blocks as normal blocks.  This
command supersedes any previous TUNNEL or CENTER commands.
\subsubsection{TUNNEL}
\begin{verbatim}
 TUNNEL {block_id} <no default>, {start} <1>, {end} <number of levels>,
      {inc} <1>
\end{verbatim}

TUNNEL defines the specified element block as a tunnel block.  A
TRANSLATE command must be in effect before this command is issued.  If a
ROTATE command is issued, all tunnel blocks are changed to normal
blocks.
For each tunnel block, a separate 3D element block is created starting
at level \{start\}, with each block having \{inc\} levels.  Any levels after
level \{end\} are put in a single block.

For example, the commands

\begin{verbatim}
        TRANSLATE 15, 15.0
        TUNNEL 999, 5, 9, 2
\end{verbatim}

create five blocks consisting of the following 3D elements (derived from
the 2D elements of element block 999):

\begin{verbatim}
        1) the elements in levels 1, 2, 3, and 4,
        2) the elements in levels 5 and 6,
        3) the elements in levels 7 and 8,
        4) the elements in level 9,
        5) the elements in levels 10, 11, 12, 13, 14, and 15.
\end{verbatim}


The block identifier of the first block is always \{block\_id\}.  The new
blocks are assigned consecutive identifiers greater than the maximum
existing (and new) identifier.
\subsubsection{CENTER}
\begin{verbatim}
 CENTER {block_id_{1}}, {block_id_{2}}, ... <all element blocks> 
\end{verbatim}

CENTER defines the specified element blocks as center blocks.  A ROTATE
command must be in effect before this command is issued.  The mesh must
be rotated a complete quadrant (90, 180, 270 or 360 degrees) and the
number of rotation levels must be a multiple of 2 for each 90 degrees of
rotation.  If em nrot is the number of rotations, there must be at least
em nrot/2 elements along the X axis in the center block.  If a TRANSLATE
command is issued, all center blocks are changed to normal blocks.  If
center blocks are defined, the center of rotation defined by the ROTATE
command is ignored.  The center of rotation is the minimum coordinate of
all elements in the center blocks.  
\subsubsection{NSETS}
\begin{verbatim}
 NSETS FRONT or BACK <no default>, {set_id_{1}}, {set_id_{2}}, ... 
      <no default>
\end{verbatim}

NSETS defines front or back node sets with the given identifiers.  The
identifiers must be unique from existing node set identifiers and
previously defined front and back node set identifiers. 

Back sets cannot be defined on a 360-degree rotation.
\subsubsection{SSETS}
\begin{verbatim}
 SSETS FRONT or BACK <no default>, {set_id_{1}}, {set_id_{2}}, ... 
      <no default>
\end{verbatim}

SSETS is equivalent to the NSETS command except that it defines side
sets.
\subsubsection{SHOW}
\begin{verbatim}
 SHOW {command} <no parameter>
\end{verbatim}

SHOW displays the settings of parameters relevant to the \{command\}.  For
example, the command SHOW BLOCK displays information about all the
element blocks.
\subsubsection{LIST}
\begin{verbatim}
 LIST VARS
\end{verbatim}

LIST VARS displays a summary of the input database.  The summary
includes the database title; the number of nodes, elements, and element
blocks; and the number of node sets and side sets.
\subsubsection{HELP}
\begin{verbatim}
 HELP {command} <no parameter>
\end{verbatim}

HELP displays information about the program command given as the
parameter.  If no parameter is given, all the command verbs are
displayed.  This command is system-dependent and may not be available on
some systems.
\subsubsection{END}
\begin{verbatim}
 END
\end{verbatim}

END ends the command input and starts the database transformation.
\subsubsection{QUIT}
QUIT ends the command input and exits the program immediately without
writing an output database. 
\subsection{Problems/Bugs}
\begin{verbatim}
 CENTER BLOCKS -- Incorrect mesh will be generated if center blocks are
       not contiguous (ie. contain slidelines).  Problem will appear
       at junction of pieces connected by slideline.  The nodes on the
        of the  piece will not be generated correctly.
\end{verbatim}

\begin{verbatim}
       FIX: Generate each piece separately and join together with GJOIN
       Problem involves changing several routines and has only been
       reported once in two years.  Sorry, Greg Sjaardema
\end{verbatim}
\subsection{ChangeLog}
\begin{verbatim}
      DATA (QAINFO(I), I=1,3) /
     * 'GEN3D   ',
     * '07/16/90',
     * 'X1.02.03' /
 C Version : Date      Time      Modifier 
 C -------   ----      ----      --------
 C X1.01.05: --/--/--  --:--:--  GREG SJAARDEMA1
 C        : Fixed orientation of NORMAL vector in GETPRO.  Added option        
 C         : DOOLDWAY to maintain compatibility with previous bug version.    
 C X1.01.05: --/--/--  --:--:--  GREG SJAARDEMA1
 C         : Initial compilation of TRANSLATE SPLINE routine                    
 C X1.01.06: --/--/--  --:--:--  GREG SJAARDEMA1
 C         : Modify SPTIN2 - total Z distance given by spline coordinates       
 C X1.01.06: --/--/--  --:--:--  GREG SJAARDEMA1
 C         : Recompile with no debug                                            
 C X1.01.06: --/--/--  --:--:--  GREG SJAARDEMA1
 C         : Fix problem in RDSPLN - increased MAXFLD to 4                      
 C X1.01.07: 03/21/90  16:23:19  GREG SJAARDEMA1                                
 C         : Reorganized SPLINE transformation, read from sys$input             
 C X1.01.08: 03/22/90  07:31:23  GREG SJAARDEMA1                                
 C         : Added A to COMAND call list to fix bug in SPLINE transformation    
 C X1.01.09: 03/22/90  08:39:58  GREG SJAARDEMA1                                
 C         : Fixed SPLXYZ - Added INIGRD lines to set ZCORD                     
 C X1.01.10: 03/22/90  09:01:10  GREG SJAARDEMA1                                
 C         : Modify SPLINE input routine (GETSPL) to use Amy's FREFLD routines  
 C X1.01.11: 03/27/90  08:44:46  GREG SJAARDEMA1                                
 C         : TRANSPLINE now reads from standard input, not file                 
 C X1.01.12: 03/27/90  12:26:46  GREG SJAARDEMA1                                
 C         : Installed as XGEN3D in XACCESS.  Recompile only                  
 C X1.01.13: 03/27/90  13:05:47  GREG SJAARDEMA1                            
 C         : Reordered SPTXYZ for efficiency                                    
 C X1.01.14: 03/29/90  15:49:40  GREG SJAARDEMA1                                
 C         : Fixed problem with 360 ROTATE - ndegr not set, X version only      
 C X1.02.00: 04/12/90  07:43:50  GREG SJAARDEMA1                                
 C         : Install SPLINE version in XACCESS. No other changes                
 C X1.02.01: 07/06/90  14:08:02  GREG SJAARDEMA2                                
 C         : Problem with noncontiguous material blocks in center             
 C         : blocks. (ie. Slidelines separating materials.)                    
 C         : Problem not fixed, error checking enabled.                        
 C         : Problem is in setting node numbers in NPCEN (sub: FNPCEN)         
 C X1.02.02: 07/09/90  14:32:11  GREG SJAARDEMA2                                
 C         : Fixed Typo in FNPCEN -- Second contiguous block warning
 C         : was IBOT, now ITOP                                                 
 C X1.02.03: 07/16/90  15:44:30  GREG SJAARDEMA4
 C         : Added SHIFT as synonym for OFFSET,
 C         : Added OFFSET/SHIFT RESET/ADD/ALL/X/Y/Z -- now same as GREPOS
\end{verbatim}

\subsection{QA}
\begin{verbatim}
       VERSION X1.02.03
\end{verbatim}

Code Goal/Status: (See HELP 1520\_CODES QA for more information)

\begin{verbatim}
          Category      Goal   Status  Comment
     -------------     ------  ------  -------
     Documentation:    Formal  Formal  See 
           Sponsor:    X       X       
      Traceability:    X       X       
    Retrievability:    X       X       
      Verification:    X       X       
          Archived:    X               
   Maintainability:    X       X       See     
        Help Entry:    X       X       
        Executable:    X       X       
\end{verbatim}

\begin{verbatim}
  Source Code Location: $CVSROOT/ACCESS/prepost/gen3d
      Archive Location: --- Not yet Archived ---
\end{verbatim}
\subsubsection{Requirements}
GEN3D Requires the following Utilities/Files:

\begin{verbatim}
       o COMMON.OLB    Library of common routines for reading/writing
                       EXODUS files and other support routines.
       o SUPES.OLB     Memory management and Free-field input routines
       o CPUIFC.OLB    Interrupt processing (Control-C).  Psuedo, non-
                       functional routine provided for non-VMS systems
\end{verbatim}

\subsection{See\_Also}
\begin{verbatim}
       o FASTQ:        Generate two-dimensional GENESIS mesh file
       o GJOIN:        Merge two or more mesh files into single file
       o GREPOS:       Reposition/Modify mesh file
       o NUMBERS:      Calculate several parameters of a mesh file
       o APREPRO:      Algebraic preprocessor for input files.
       o BLOT:         Plot mesh files.
\end{verbatim}


\end{document}
